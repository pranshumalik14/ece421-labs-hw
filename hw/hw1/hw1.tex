\documentclass{article}
\usepackage{amsmath}
\usepackage{amsfonts}
\usepackage{amssymb}
\usepackage{amsthm}
\usepackage{fancyhdr}
\usepackage{hyperref}
\usepackage[top=1.2in, textwidth=6.5in]{geometry}

% set up margins
\textheight=9in
\parskip=.07in
\parindent=0in
\headheight=23pt
\pagestyle{fancy}

% set up header
\newcommand{\setheader}[5]
{
	\lhead{{\sc #1}\\{\sc #2} ({\small \it #3})}
	\rhead{{\bf #4}\\{#5}}
}

% set up some shortcuts
\newcommand{\R}{\mathbb{R}}
\newcommand{\N}{\mathbb{N}}
\newcommand{\Z}{\mathbb{Z}}
\newcommand{\mat}[1]{\begin{bmatrix}#1\end{bmatrix}}
\renewcommand{\vec}[1]{\mathbf{#1}}

\begin{document}
    \setheader{ECE421: Introduction to Machine Learning}
    {Homework Assignment 1}{\today}{Pranshu Malik}{1004138916}

    \textbf{Solution 1: Perceptron Learning Algorithm}
	
    Given a dataset $\mathcal{D} = \{(\vec{x}_i, y_i)\}_{i=1}^N$, where 
    $\vec{x}_i\in\R^{d+1}$ is an augmented datavector 
    $\mat{1 & x_{i,1} & \ldots & x_{i, d}}^\top$ with label $y_i \in 
    \{-1,1\}$, the perceptron learning algorithm performs linear 
    classification that can be written as
    \begin{equation}
       \hat{y}_i = h(\vec{x}_i) = \text{sign}(\vec{w}^\top\vec{x}_i),
    \end{equation}

    where the model parameter $\vec{w} \in \R^{d+1}$ is a weightvector and
    $\hat{y}_n$ is the predicted label. Given 
    an incorrectly classified training point $(\vec{x}_n,y_n)\in\mathcal{D}$, 
    the weight update rule is
    \begin{equation}
        \label{eqn:update_wgt}
        \vec{w}^\prime = \vec{w} + y_n\vec{x}_n.
    \end{equation}

    \begin{proof}[\unskip\nopunct]
        a) For a misclassified training point,
        let $\text{sign}(\vec{w}^\top\vec{x}_n) = \hat{y}_n$, for
        which we necessarily have $\hat{y}_n = -y_n \neq y_n$, and thus 
        $y_n\hat{y}_n = -1$. Therefore, in general, without the $\text{sign}
        (\cdot)$ function that strips the magnitude of the dot product, 
        we will have $y_n\vec{w}^\top\vec{x}_n < 0$.
    \end{proof}

    \begin{proof}[\unskip\nopunct]
        b) From the weight update step in equation \eqref{eqn:update_wgt}, we 
        have 
        \[
            y_n\vec{w}^{\prime\top}\vec{x}_n = y_n(\vec{w}^\top + 
            y_i\vec{x}_n^\top)\vec{x}_n = y_n\vec{w}^\top\vec{x}_n+ 
            y_n^2||\vec{x}_n||^2. 
        \]
        Since $y_n^2||\vec{x}_n||^2 > 0$, we always have
        $y_n\vec{w}^{\prime\top}\vec{x}_n > 
        y_n\vec{w}^{\top}\vec{x}_n$.
    \end{proof}

    \begin{proof}[\unskip\nopunct]
        c) From the above two proofs, notice that a misclassification always 
        results in $y_n\vec{w}^\top \vec{x}_n < 0$, whereas 
        a correct classification yields $y_n\vec{w}^\top \vec{x}_n
        > 0$. Therefore, it can be seen that the update step introduces a net
        addition towards making the inner product positive (hence a correct 
        classification) for a misclassified point. 
        This can also be viewed geometrically, where
        the weightvector $\vec{w}$, being normal to the decision boundary (a 
        hyperplane), changes its alignment to the misclassified datavector 
        $\vec{x}_n$. In the case where $y_n = 1 = -\hat{y}_n$, we have an 
        obtuse angle between the  weight and data vectors (since $\vec
        {w}^\top \vec{x}_n < 0$), for which $\vec{w} ^{\prime} = \vec{w}+
        \vec{x}_n$ aligns more closely with $\vec{x}_n$. In the other case 
        where $y_n = -1 = -\hat{y}_n$, we have an acute angle between the two
        vectors, and the updated weight $\vec{w}^{\prime} =
        \vec{w} - \vec{x}_n$ aligns less with $\vec{x}_n$. In both cases, the
        update magnitude is proportional to the norm of the datavector and 
        is made in the correct direction, though it may misclassify other 
        datavectors or not completely resolve the current error in one go.
        But since there is no selection preference or position constraint for 
        a misclassified datavector, the algorithm will continue to explore 
        freely in each update.
    \end{proof}

    Thus, with the above reasoning, we can conclude that we will have a
    definite and correct termination of the perceptron algorithm for a 
    linearly separable dataset, since we will eventually come across an
    update step that resolves the only remaining misclassification when the
    decision boundary places itself in the label-separation region. Note that
    this algorithm does not produce unique (or deterministic) results and 
    will not terminate for non-linearly separable datasets.
    
    \vspace{0.5cm}

    \textbf{Solution 2: Linear Regression}

    Given a dataset $\mathcal{D} = \{(\vec{x}_i, y_i)\}_{i=1}^N$, where 
    $\vec{x}_i\in\R^{d+1}$ is an augmented datavector 
    $\mat{1 & x_{i,1} & \ldots & x_{i, d}}^\top$ with the corresponding 
    output $y_i \in \R$, the linear regression model can be written as
    \begin{equation}
       \hat{y}_i = h(\vec{x}_i) = \vec{w}^\top\vec{x}_i,
    \end{equation}
    where the model parameter $\vec{w} \in \R^{d+1}$ is a weightvector that
    is associated with the in-sample error, or loss function, being the 
    mean squared error over the training dataset
    \begin{equation}
        \label{eqn:loss_fun_regression}
        E_{\text{in}}(\vec{w}) = \frac{1}{N}\sum_{i=1}^N (y_i - \hat{y}_i)^2 
        = \frac{1}{N}||\vec{y}-\hat{\vec{y}}||^2.
    \end{equation}

    In equation \eqref{eqn:loss_fun_regression}, the vectors $\vec{y} = 
    \mat{y_1 & \ldots & y_N}^\top$ and $\hat{\vec{y}} = \mat{\hat{y}_1 & 
    \ldots & \hat{y}_N}$ are the true and predicted outputs over the training 
    dataset, respectively. Note that we can write $\hat{\vec{y}} = \vec{X}
    \vec{w}$, for $\vec{X}=\mat{\vec{x}_1^\top & \ldots & \vec{x}_N^
    \top} ^\top \in \R^{N \times d+1}$ since the inner product is 
    commutative, i.e., $\vec{w}^\top \vec{x} = \vec{x}^\top \vec{w}$.

    \begin{proof}[\unskip\nopunct]
        a) In general, the squared norm for a vector is given by the inner 
        product with itself. Therefore, we can write
        $\frac{1}{N}||\vec{y}-\vec{X}\vec{w}||^2 = \frac{1}{N}(\vec{y}-\vec{X}
        \vec{w})^\top(\vec{y} - \vec{X}\vec{w})$, which upon expanding 
        gives the scalar loss 
        \begin{align}
            E_{\text{in}}(\vec{w}) &= 
            \frac{1}{N}\Big(\vec{y}^\top\vec{y}-(\vec{X}
            \vec{w})^\top\vec{y} - \vec{y}^\top \vec{X}\vec{w}
            + (\vec{X}\vec{w})^\top (\vec{X}\vec{w}) \Big) \nonumber\\
            &= \frac{1}{N}\big(\vec{y}^\top\vec{y} - 2\vec{y}^\top 
            \vec{X}\vec{w} + \vec{w}^\top \vec{X}^\top
            \vec{X}\vec{w}\big) \label{eqn:loss_fun_regression_vector}\\
            &= \frac{1}{N}\big(||\vec{y}||^2-2\vec{y}^\top\hat{\vec{y}} 
            + ||\hat{\vec{y}}||^2\big) \nonumber
        \end{align}
        Note that $\vec{y}^\top\vec{X}\vec{w}$ and $(\vec{X}\vec{w})
        ^\top\vec{y}$ are equal and were grouped together since 
        they are scalar quantities.
    \end{proof}

    \begin{proof}[\unskip\nopunct]
        b) To verify the gradient $\nabla_\vec{w}E_{\text{in}}(\vec{w})$ for
        both the sum and matrix expressions shown above, we first consider
        the gradient of error summation in equation 
        \eqref{eqn:loss_fun_regression}
        \begin{align*}
            &\nabla_{\vec{w}}\left(\frac{1}{N}\sum_{i=1}^N (y_i - 
            \underbrace{\vec{w}^\top\vec{x}_i}_{\hat{\vec{y}}_i})^2
            \right) = \frac{1}{N}\mat{\frac{\partial}{\partial w_0}
            \sum_{i=1}^N\big(y_i - (w_0x_{i, 0}+\ldots+ w_dx_{i,d})\big)^2 
            \\\vdots\\\frac{\partial}{\partial w_d}\sum_{i=1}^N\big(y_i- 
            (w_0x_{i, 0}+\ldots+w_dx_{i,d})\big)^2} = 
            \frac{2}{N}\mat{\sum_{i=1}^N(\vec{w}^\top\vec{x}_i-y_i)x_{i,0}
            \\\vdots\\\sum_{i=1}^N (\vec{w}^\top\vec{x}_i-y_i)x_{i,d}}\\
            &= \frac{2}{N}\mat{\sum_{i=1}^Nx_{i,0}(\vec{x}_i^\top\vec{w})-
            \sum_{i=1}^Nx_{i,0}y_i\\\vdots\\\sum_{i=1}^Nx_{i,d}(\vec{x}
            _i^\top\vec{w})-\sum_{i=1}^Nx_{i,d}y_i} = \frac{2}{N}\left(
            \mat{\sum_{i=1}^Nx_{i,0}\vec{x}_i^\top\\\vdots\\\sum_{i=1}^N
            x_{i,d}\vec{x}_i^\top}\vec{w}-\sum_{i=1}^Ny_i\vec{x}_i\right) = 
            \frac{2}{N}\big(\vec{X}^\top\vec{X}\vec{w}-\vec{X}^\top\vec{y}
            \big),
        \end{align*}
        and then consider the gradient of the matrix expression in equation 
        \eqref{eqn:loss_fun_regression_vector},
        \[
            \nabla_{\vec{w}}\left(\frac{1}{N}\big(\vec{y}^\top\vec{y} - 
            2\underbrace{\vec{y}^\top \vec{X}}_{\vec{v}^\top}
            \vec{w} + \vec{w}^\top\underbrace{\vec{X}^\top\vec{X}}
            _{\vec{A}}\vec{w}\big)
            \right) = \frac{1}{N}\big(-2\vec{v}+ 2\vec{A}\vec{w}\big) = 
            \frac{2}{N}\big(\underbrace{\vec{X}^\top\vec{X}}_{\vec{A}=\vec{A}
            ^\top}\vec{w} - \vec{X}^\top\vec{y}\big).
        \]
        Hence, we have shown that both expressions are equivalent.
    \end{proof}

    \begin{proof}[\unskip\nopunct]
        c) Assuming $\vec{A} = \vec{X}^\top\vec{X}$ to be invertible, we let 
        $\vec{w}^\star=\vec{A}^{-1}\vec{X}^\top\vec{y}$ and 
        define $\hat{\vec{y}}_{\text{ls}}=\vec{X}\vec{w}^\star$. 
        Based on these quantities, we can rewrite the loss function as
        \begin{align*}
            E_{\text{in}}(\vec{w}) &= \frac{1}{N}||\underbrace{(\hat{\vec{y}}_
            {\text{ls}}-\hat{\vec{y}})+(\vec{y}-\hat{\vec{y}}_{\text{ls}})}_
            {=\vec{y}-\hat{\vec{y}}}||^2 = \frac{1}{N}\big((\hat{\vec{y}}_
            {\text{ls}}-\hat{\vec{y}})^\top+(\vec{y}-\hat{\vec{y}}_
            {\text{ls}})^\top\big)\big((\hat{\vec{y}}_{\text{ls}}-\hat{
            \vec{y}})+(\vec{y}-\hat{\vec{y}}_{\text{ls}})\big)\\
            &= \frac{1}{N}\big(||\hat{\vec{y}}_{\text{ls}}-\hat{\vec{y}}||^2
            +2(\hat{\vec{y}}_{\text{ls}}-\hat{\vec{y}})^\top(\vec{y}-\hat{
            \vec{y}}_{\text{ls}})+||\vec{y}-\hat{\vec{y}}_
            {\text{ls}}||^2\big), \text{ where }\hat{\vec{y}}=\vec{X}\vec{w}.
        \end{align*}
        Further evaluating the inner product, we get
        \begin{align*}
            (\hat{\vec{y}}_{\text{ls}}-\hat{\vec{y}})^\top(\vec{y}-\hat{
            \vec{y}}_{\text{ls}}) &= (\vec{X}\vec{w}^\star-\vec{X}\vec{w})^
            \top(\vec{y}-\vec{X}\vec{w}^\star)=(\vec{w}^\star-\vec{w})^\top(
            \vec{X}^\top\vec{y}-\vec{X}^\top\vec{X}\vec{w}^\star)\\
            &= (\vec{w}^\star-\vec{w})^\top(\vec{X}^\top\vec{y}-\vec{A}\vec{A}
            ^{-1}\vec{X}^\top\vec{y}) = 0.
        \end{align*}
        Therefore, we have shown that 
        \begin{equation}
            \label{eqn:loss_least_squares}
            E_{\text{in}}(\vec{w}) = \frac{1}{N}\big(||\hat{\vec{y}}_{
            \text{ls}}-\vec{X}\vec{w}||^2+||\vec{y}-\hat{\vec{y}}_{\text{ls}}
            ||^2\big).
        \end{equation}
        Note that, this result follows from setting the gradient in part b) 
        to zero, from which we can infer $\hat{\vec{y}}_{\text{ls}}$ to be 
        the least squares prediction for the training dataset outputs 
        $\vec{y}$.
    \end{proof}

    \begin{proof}[\unskip\nopunct]
        d) From equation \eqref{eqn:loss_least_squares}, it can be seen that 
        the loss is minimum when the predicted output $\hat{\vec{y}}$ equals 
        least squares prediction 
        $\hat{\vec{y}}_{\text{ls}}$, i.e.,
        \[
            \min_{\vec{w}} E_{\text{in}}(\vec{w})=\min_{\vec{w}}\frac{1}{N}
            \big(\underbrace{||\hat{\vec{y}}_{\text{ls}}-\vec{X}\vec{w}||^2}_
            {\text{variable}} + \underbrace{||\vec{y}-\hat{\vec{y}}_
            {\text{ls}}||^2}_{\text{constant}}\big) = \frac{1}{N}||\vec{y}
            -\hat{\vec{y}}_{\text{ls}}||^2,
        \]
        since the variable term goes to zero (minimum norm). It follows from 
        the definition of $\hat{\vec{y}}_{\text{ls}}$ in the previous proof, 
        that the minimizer for the loss function is $\vec{w}^\star 
        = \vec{A}^{-1}\vec{X}^\top\vec{y}$. Geometrically, 
        $\hat{\vec{y}}_{\text{ls}}=\vec{X}\vec{w}^\star = \vec{P}\vec{y}$ 
        corresponds to the projection of $\vec{y}$ onto the hyperplane 
        $\mathcal{P} = \text{Im}(X)$ since it is given by a linear 
        combination of the column vectors of $\vec{X}$. This also means that
        the residual error $(\vec{y}-\hat{\vec{y}}_{\text{ls}})$ is 
        orthogonal to $\mathcal{P}$ which was supported by the previous proof 
        when we showed that $\vec{X}^\top(\vec{y}-\vec{X}\vec{w}^\star) = 0$. 
        Therefore, we have shown that any prediction by the linear regression 
        model $\hat{\vec{y}}=\vec{X} \vec{w}\in\mathcal{P}$, from which it 
        follows that $(\hat{\vec{y}}-\hat{\vec{y}}_{\text{ls}})\perp
        (\vec{y}-\hat{\vec{y}}_{\text{ls}})$
    \end{proof}

\end{document}