\documentclass{article}
\usepackage{amsmath}
\usepackage{amsfonts}
\usepackage{amssymb}
\usepackage{amsthm}
\usepackage{fancyhdr}
\usepackage{hyperref}
\usepackage[top=1.2in, textwidth=6.5in]{geometry}

% set up margins
\textheight=9in
\parskip=.07in
\parindent=0in
\headheight=23pt
\pagestyle{fancy}

% set up header
\newcommand{\setheader}[5]
{
	\lhead{{\sc #1}\\{\sc #2} ({\small \it #3})}
	\rhead{{\bf #4}\\{#5}}
}

% set up some shortcuts
\newcommand{\R}{\mathbb{R}}
\newcommand{\N}{\mathbb{N}}
\newcommand{\Z}{\mathbb{Z}}
\newcommand{\mat}[1]{\begin{bmatrix}#1\end{bmatrix}}
\renewcommand{\vec}[1]{\mathbf{#1}}

\begin{document}
    \setheader{ECE421: Introduction to Machine Learning}
    {Homework Assignment 7}{\today}{Pranshu Malik}{1004138916}

    \textbf{Solution 1: Gaussian Mixture Models (Final Exam 2018, Problem 5)}

    We are given a pre-trained Gaussian Mixture Model (GMM) for the marks of 
    $K=2$ classes: undergraduate students $(w_1=\frac{2}{3}, \mu_1=70, 
    \sigma_1^2=100)$ and graduate students $(w_2=\frac{1}{3}, \mu_2=80, 
    \sigma_2^2=25)$.

    a) Taking a random variable (RV) $X$, with the probability distribution 
    defined by the GMM, the probability $P(X\geq80) = P(X\geq80\mid\text
    {undergraduate})P(\text{undergraduate}) + P(X\geq80\mid\text{graduate})P
    (\text{graduate})$ by the law of total probability. The weights of the 
    GMM convey the share of total population for each class, and hence $P
    (\text{undergraduate})=w_1$ and $P(\text{graduate})=w_2$. Furthermore, 
    for an RV $W\sim\mathcal{N}(\mu_W, \sigma_W^2)$ we can approximate the 
    probability $P(|W-\mu_W|\leq\sigma_W)=\frac{2}{3}$, using which we get $P
    (X\geq80) = (\frac{1}{2}-\frac{2}{3}\frac{1}{2})\frac{2}{3} + \frac{1}{2}
    \frac{1}{3} = \frac{5}{18}$.

    b) The probability $P(\text{undergraduate}\mid X\geq80) = \frac{P(\text
    {undergraduate}, X\geq80)}{P(X\geq80)} = \frac{P(X\geq80\mid\text
    {undergraduate})P(\text{undergraduate})}{P(X\geq80)}$, using Bayes' 
    theorem. Then, using the previous part, we get $P(\text{undergraduate}
    \mid X\geq80) = \frac{(\frac{1}{2}-\frac{2}{3}\frac{1}{2})\frac{2}{3}}
    {\frac{5}{18}} = \frac{2}{5}$.

\end{document}