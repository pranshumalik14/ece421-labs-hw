\documentclass{article}
\usepackage{amsmath}
\usepackage{amsfonts}
\usepackage{amssymb}
\usepackage{amsthm}
\usepackage{fancyhdr}
\usepackage{hyperref}
\usepackage[top=1.2in, textwidth=6.5in]{geometry}

% set up margins
\textheight=9in
\parskip=.07in
\parindent=0in
\headheight=23pt
\pagestyle{fancy}

% set up header
\newcommand{\setheader}[5]
{
	\lhead{{\sc #1}\\{\sc #2} ({\small \it #3})}
	\rhead{{\bf #4}\\{#5}}
}

% set up some shortcuts
\newcommand{\R}{\mathbb{R}}
\newcommand{\N}{\mathbb{N}}
\newcommand{\Z}{\mathbb{Z}}
\newcommand{\mat}[1]{\begin{bmatrix}#1\end{bmatrix}}
\renewcommand{\vec}[1]{\mathbf{#1}}

\begin{document}
    \setheader{ECE421: Introduction to Machine Learning}
    {Homework Assignment 2}{\today}{Pranshu Malik}{1004138916}

    \textbf{Solution 1: Gradient Computation}
    
    i) 


    ii)


    iii)


    iv)

    \vspace{0.5cm}

    \textbf{Solution 2: Logistic Regression}

    \begin{proof}[\unskip\nopunct]
        a)
    \end{proof}

    \begin{proof}[\unskip\nopunct]
        b)
    \end{proof}

    \begin{proof}[\unskip\nopunct]
        c)
    \end{proof}

    \begin{proof}[\unskip\nopunct]
        d)
    \end{proof}

    \vspace{0.5cm}

    \textbf{Solution 3: Midterm 2017, Problem 4}

    a)

    b)

\end{document}