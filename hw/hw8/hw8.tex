\documentclass{article}
\usepackage{amsmath}
\usepackage{amsfonts}
\usepackage{amssymb}
\usepackage{amsthm}
\usepackage{fancyhdr}
\usepackage{hyperref}
\usepackage[top=1.2in, textwidth=6.5in]{geometry}

% set up margins
\textheight=9in
\parskip=.07in
\parindent=0in
\headheight=23pt
\pagestyle{fancy}

% set up header
\newcommand{\setheader}[5]
{
	\lhead{{\sc #1}\\{\sc #2} ({\small \it #3})}
	\rhead{{\bf #4}\\{#5}}
}

% set up some shortcuts
\newcommand{\R}{\mathbb{R}}
\newcommand{\N}{\mathbb{N}}
\newcommand{\Z}{\mathbb{Z}}
\newcommand{\mat}[1]{\begin{bmatrix}#1\end{bmatrix}}
\renewcommand{\vec}[1]{\mathbf{#1}}

\begin{document}
    \setheader{ECE421: Introduction to Machine Learning}
    {Homework Assignment 8}{\today}{Pranshu Malik}{1004138916}

    \textbf{Solution 1: Gaussian Mixture Models (Final Exam 2019, Problem 3)}

    The proposed Gaussian Mixture Model (GMM) predicts the probability that a
    lake is poisonous. The data with pH level, $l_i$, of each lake is 
    given in ascending order, i.e., $\mathcal{D} = \{l_i\}_{i=1}^{N}, 
    \text{ such that } l_i \leq l_j \text{ for } i < j$. By the problem 
    definition, we have $K=2$ classes, with the first class representing the 
    poisonous lakes, and they have their respective mean and variance values, 
    $\mu_i, \sigma^2_i$ for $i = 1, 2$, where it is hypothesized that $\mu_1 
    \geq \mu_2$. Furthermore, according to the overall split of the two 
    classes amongst all lakes, their weights (or probability of selection 
    are) are $p_1$ and $p_2$, respectively.

    a) The probability that a random lake is poisonous given its pH level, 
    $l$, can be written as
    \begin{align*}
        P(\text{poisonous}\mid l) &= \frac{f_{\text{pH}\mid\text{lake}}
        (l\mid\text{poisonous})P(\text{poisonous})}{f_{\text{pH}}(l)}\\
        &= \frac{\mathcal{N}(\mu_1, \sigma_1^2)p_1}{\mathcal{N}(\mu_1, 
        \sigma_1^2)p_1 + \mathcal{N}(\mu_2, \sigma_2^2)p_2}
    \end{align*}

    b) The pseudocode for EM algorithm for training our GMM with hard 
    decisions is given below. The initial clusters are a split of the  
    dataset at index $k$, i.e., $\mathcal{B}_1[0] = \{l_{k+1}, \ldots, l_N\}$ 
    and $\mathcal{B}_2[0] = \{l_1, \ldots, l_k\}$.
    \begin{itemize}
        \item \textbf{Do}: 
        \begin{itemize}
            \item[$\bullet$] update $p_i, \mu_i, \sigma_i^2$, for $i=1,2$
            \begin{itemize}
                \item[$\circ$] $p_1 = \frac{|\mathcal{B}_1|}{N}$ and $p_2 = \frac
                {|\mathcal{B}_2|}{N}$
                \item[$\circ$] $\mu_1 = \frac{1}{|\mathcal{B}_1[n]|}\sum_
                {l_i\in\mathcal{B}_1[n]}l_i$ and $\mu_2 = \frac{1}{|\mathcal{B}_2[n]|
                }\sum_{l_i\in\mathcal{B}_2[n]}l_i$
                \item[$\circ$] $\sigma_1^2 = \frac{1}{|\mathcal{B}_1[n]|}\sum_
                {l_i\in\mathcal{B}_1[n]}{(l_i - \mu_1)^2}$ and $\sigma_2^2 = \frac{1}
                {|\mathcal{B}_2[n]|}\sum_{l_i\in\mathcal{B}_2[n]}{(l_i - \mu_2)^2}$ 
            \end{itemize}
            \item[$\bullet$] 
        \end{itemize}
        \item \textbf{While}: $\mathcal{B}_1[n] \neq \mathcal{B}_1[n+1]$ and 
        $\mathcal{B}_2[n] \neq \mathcal{B}_2[n+1]$.
    \end{itemize}

    c)

    d)

    \vspace{0.5cm}

    \textbf{Solution 2: }

    Hello again

\end{document}