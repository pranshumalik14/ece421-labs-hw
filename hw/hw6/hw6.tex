\documentclass{article}
\usepackage{amsmath}
\usepackage{amsfonts}
\usepackage{amssymb}
\usepackage{amsthm}
\usepackage{fancyhdr}
\usepackage{hyperref}
\usepackage[top=1.2in, textwidth=6.5in]{geometry}

% set up margins
\textheight=9in
\parskip=.07in
\parindent=0in
\headheight=23pt
\pagestyle{fancy}

% set up header
\newcommand{\setheader}[5]
{
	\lhead{{\sc #1}\\{\sc #2} ({\small \it #3})}
	\rhead{{\bf #4}\\{#5}}
}

% set up some shortcuts
\newcommand{\R}{\mathbb{R}}
\newcommand{\N}{\mathbb{N}}
\newcommand{\Z}{\mathbb{Z}}
\newcommand{\mat}[1]{\begin{bmatrix}#1\end{bmatrix}}
\renewcommand{\vec}[1]{\mathbf{#1}}

\begin{document}
    \setheader{ECE421: Introduction to Machine Learning}
    {Homework Assignment 6}{\today}{Pranshu Malik}{1004138916}

    \textbf{Solution 1: K-Means Clustering (Final Exam 2018, Problem 6)}

    Given the dataset $\mathcal{D} = \{0, 0.5, 0.5+\Delta, 1.5+\Delta\}$,
    where $\Delta \geq 0$ and considering $K=2$ clusters, $\mathcal{B}_1$ and 
    $\mathcal{B}_2$, are the two clusters with means $\mu_1$ and $\mu_2$ 
    respectively, we solve the problem subparts using Lloyd's algorithm.

    a) Letting $\Delta=0.5$ and initializing the means to $\mu_1[0]=1, 
    \mu_2[0] = 2$, we find the following clustering:
    \begin{itemize}
        \item $\mathcal{B}_1[0] = \{0, 0.5, 1\}$
        \item $\mathcal{B}_2[0] = \{2\}$.
    \end{itemize}
    Continuing the Lloyd's algorithm, we find the new means:
    \begin{itemize}
        \item $\mu_1[1] = \frac{0+0.5+1}{3}=0.5$
        \item $\mu_2[1] = 2$.
    \end{itemize}
    Following the changed means, we again find the new clustering:
    \begin{itemize}
        \item $\mathcal{B}_1[1] = \{0, 0.5, 1\}$
        \item $\mathcal{B}_2[1] = \{2\}$.
    \end{itemize}
    Since the cluster memberships have converged, we stop.

    b) 
    
\end{document}