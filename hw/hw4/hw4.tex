\documentclass{article}
\usepackage{amsmath}
\usepackage{amsfonts}
\usepackage{amssymb}
\usepackage{amsthm}
\usepackage{fancyhdr}
\usepackage{hyperref}
\usepackage[top=1.2in, textwidth=6.5in]{geometry}
\usepackage{tikzit}
\input{nn_nodes.tikzstyles}

% set up margins
\textheight=9in
\parskip=.07in
\parindent=0in
\headheight=23pt
\pagestyle{fancy}

% set up header
\newcommand{\setheader}[5]
{
	\lhead{{\sc #1}\\{\sc #2} ({\small \it #3})}
	\rhead{{\bf #4}\\{#5}}
}

% set up some shortcuts
\newcommand{\R}{\mathbb{R}}
\newcommand{\N}{\mathbb{N}}
\newcommand{\Z}{\mathbb{Z}}
\newcommand{\mat}[1]{\begin{bmatrix}#1\end{bmatrix}}
\renewcommand{\vec}[1]{\mathbf{#1}}

\begin{document}
    \setheader{ECE421: Introduction to Machine Learning}
    {Homework Assignment 4}{\today}{Pranshu Malik}{1004138916}

    \textbf{Solution 1: <>}

    Hello

    \vspace{0.5cm}

    \textbf{Solution 2: <>}

    Hello again.
    \ctikzfig{OR_nn}
    \ctikzfig{AND_nn}
    \ctikzfig{innerprod_nn}
    \ctikzfig{OR_x1_x2bar_x3}

\end{document}