\documentclass{article}
\usepackage{amsmath}
\usepackage{amsfonts}
\usepackage{amssymb}
\usepackage{amsthm}
\usepackage{fancyhdr}
\usepackage{hyperref}
\usepackage[top=1.2in, textwidth=6.5in]{geometry}

% set up graphics
\usepackage{tikzit}
\input{nn_nodes.tikzstyles}
\usepackage{pgfplots}
\pgfplotsset{compat=1.9}


% set up margins
\textheight=9in
\parskip=.07in
\parindent=0in
\headheight=23pt
\pagestyle{fancy}

% set up header
\newcommand{\setheader}[5]
{
	\lhead{{\sc #1}\\{\sc #2} ({\small \it #3})}
	\rhead{{\bf #4}\\{#5}}
}

% set up some shortcuts
\newcommand{\R}{\mathbb{R}}
\newcommand{\N}{\mathbb{N}}
\newcommand{\Z}{\mathbb{Z}}
\newcommand{\mat}[1]{\begin{bmatrix}#1\end{bmatrix}}
\renewcommand{\vec}[1]{\mathbf{#1}}

\begin{document}
    \setheader{ECE421: Introduction to Machine Learning}
    {Homework Assignment 4}{\today}{Pranshu Malik}{1004138916}

    \textbf{Solution 1: Decision Functions}

    We would like to implement a piecewise linear target 
    (ground-truth) function $f$ which classifies regions in $\R^2$ into two 
    classes, positive $(+)$ and negative $(-)$, using three independent
    hypothesis functions: $h_1$, $h_2$, $h_3$.

    \begin{proof}[\unskip\nopunct]
        a) From the figure, it can be seen that there are three regions with 
        the positive label, namely $\mathcal{R}_1 = \overline{h_1}h_2h_3$, 
        $\mathcal{R}_2 = h_1h_2\overline{h_3}$, and $\mathcal{R}_3 = 
        h_1\overline{h_2}h_3$. Therefore, for any given input vector $\vec{x} 
        = \mat{x_1 & x_2}^\top$, $f(\vec{x}) = +$ if and only if $\vec{x} \in 
        \mathcal{R}_1\cup\mathcal{R}_2\cup\mathcal{R}_3$, or equivalently, we 
        can write the resulting expression for $f$ by using the 
        sum of products (SOP) criterion as follows: $f = \overline{h_1}h_2h_3 
        + h_1h_2\overline{h_3} + h_1\overline{h_2}h_3$.
    \end{proof}

    b) Yes, similar to above, we can decompose any target decision function 
    using the SOP criterion over all $+$ regions. Furthermore, the Universal 
    Approximation Theorem (for Neural Networks) states that any smooth 
    decision boundary can be approximated by piecewise linear functions which 
    can always be implemented using three layers (including input and output 
    layers) and sufficiently many nodes in the hidden layer.

    \vspace{0.5cm}

    \textbf{Solution 2: Decision Function Implementation}

    a) To extend the logical \texttt{OR} function for 
    vectors $\vec{x}\in\{-1,1\}^M$, we observe the following: the output is
    $-1$ if and only if all components of the input are $-1$, i.e. 
    $\sum_{i=1}^{M}x_i = -M$; otherwise the output remains $+1$. Therefore,
    the decision rule can be written as $h(\vec{x}) = \text{sign}\left(\sum_
    {i=1}^M x_i + M - \frac{1}{2}\right)$. In all graphical representations
    below, the activation function $\psi$ is the $\text{sign}$ function.
    \begin{figure}[!h]
        \ctikzfig{OR_nn}
        \caption{Neural network implementation of logical \texttt{OR}}
    \end{figure}
    
    Likewise, the decision rule for \texttt{AND} operation, can be written as
    $h(\vec{x}) = \text{sign}\left(\sum_{i=1}^M x_i - M + \frac{1}{2}\right)$.
    Its graphical representation is shown below.
    \begin{figure}[!h]
        \ctikzfig{AND_nn}
        \caption{Neural network implementation of logical \texttt{AND}}
    \end{figure}
    
    \pagebreak

    b) Following the above, we can implement the inner product of a vector 
    $\vec{x}$ and weights $\vec{w}$ by the network given below.
    \begin{figure}[!ht]
        \ctikzfig{innerprod_nn}
        \caption{Neural network implementation of $h(\vec{x}) = \text{sign}
        (\vec{w}^\top\vec{x})$ for $\vec{x}, \vec{w} \in \R^n$}
    \end{figure}
    
    c) Again, following part a), we can implement \texttt{OR} with negated 
    arguments as below (note $M=3$). 
    \begin{figure}[!ht]
        \ctikzfig{OR_x1_x2bar_x3}
        \caption{Neural network implementation of $\texttt{OR}(x_1,\overline
        {x_2},x_3)$}
    \end{figure}

    \pgfplotsset{custom/.append style={axis x line=middle, 
    axis y line=middle, xlabel={$x_1$}, ylabel={$x_2$}, axis 
    equal}}

    \vspace{0.5cm}

    \textbf{Solution 3: Decision Function Implementation (Midterm 2019, 
    Problem 3)}

    a) We get the following truth table after evaluating all combinations of
    inputs $(x_1, x_2) \in \{0,1\}^2$.
    
    \begin{table}[h]
        \[
            \begin{array}{cccc}
                x_1 & x_2 & \nu & \psi(\nu) \\ \hline
                0 & 0 & 0 & 0 \\
                0 & 1 & 1 & 0 \\
                1 & 0 & 1 & 0 \\
                1 & 1 & 2 & 1
            \end{array}
        \]
    \end{table}

    By observation, it is evident that the function is the \texttt{AND} 
    operation.

    b) Again, by observing the given truth table, it is evident that the new 
    function is the \texttt{OR} operation which can be realized by the same 
    neural network in part a), but by modifying of the activation 
    function to be 

    \[
        \psi(\nu) = 
        \begin{cases}
            1 & \nu \geq 1 \\
            0 & \text{else }
        \end{cases}.
    \]

    c) The given truth table is shown in graphical form in Figure 
    \ref{fig:XOR}, from which it is evident that the target function (\texttt
    {XOR}) is not separable by a single linear function, or plane, and hence 
    can not be implemented using a single layer. But, the target labels can 
    be computed by one layer if we can allow bivariate non-linear 
    activation functions, such as
    \[
        \psi(x_1, x_2) = 
        \begin{cases}
            1 & \frac{1}{2} \leq x_1 + x_2 \leq \frac{3}{2} \\
            0 & \text{else}
        \end{cases}.
    \]

    \begin{figure}[ht]
        \centering
        \begin{tikzpicture}
            \begin{axis}[custom,xmin=-0.1,xmax=1.25,ymin=-0.15, 
            ymax=1.25,xtick={0,1},ytick={0,1}, legend pos=outer north east]
            \addplot[mark=*,fill=white,only marks,mark size=4pt] 
            coordinates {(0,0)(1,1)};
            \addlegendentry{$g(\vec{x}) = 0$}
            \addplot[mark=*,only marks,mark size=4pt] 
            coordinates {(0,1)(1,0)};
            \addlegendentry{$g(\vec{x}) = 1$}
            \addplot[dashed] {-x+0.5};
            \addplot[dashed] {-x+1.5};
            \addlegendentry{$\psi(\vec{x})$}
            \end{axis}
        \end{tikzpicture}
        \caption{$\texttt{XOR}(\vec{x})$ for $\vec{x} \in \{0,1\}^2$}
        \label{fig:XOR}
    \end{figure}

\end{document}