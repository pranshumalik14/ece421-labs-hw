\documentclass{article}
\usepackage{amsmath}
\usepackage{amsfonts}
\usepackage{amssymb}
\usepackage{amsthm}
\usepackage{fancyhdr}
\usepackage{hyperref}
\usepackage[top=1.2in, textwidth=6.5in]{geometry}

% set up graphics
\usepackage{tikzit}
\input{nn_nodes.tikzstyles}
\usepackage{pgfplots}
\pgfplotsset{compat=1.9}


% set up margins
\textheight=9in
\parskip=.07in
\parindent=0in
\headheight=23pt
\pagestyle{fancy}

% set up header
\newcommand{\setheader}[5]
{
	\lhead{{\sc #1}\\{\sc #2} ({\small \it #3})}
	\rhead{{\bf #4}\\{#5}}
}

% set up some shortcuts
\newcommand{\R}{\mathbb{R}}
\newcommand{\N}{\mathbb{N}}
\newcommand{\Z}{\mathbb{Z}}
\newcommand{\mat}[1]{\begin{bmatrix}#1\end{bmatrix}}
\renewcommand{\vec}[1]{\mathbf{#1}}

\begin{document}
    \setheader{ECE421: Introduction to Machine Learning}
    {Homework Assignment 4}{\today}{Pranshu Malik}{1004138916}

    \textbf{Solution 1: <>}

    Hello

    \vspace{0.5cm}

    \textbf{Solution 2: <>}

    Hello again.
    \ctikzfig{OR_nn}
    \ctikzfig{AND_nn}
    \ctikzfig{innerprod_nn}
    \ctikzfig{OR_x1_x2bar_x3}

    \pgfplotsset{custom/.append style={axis x line=middle, 
    axis y line=middle, xlabel={$x_1$}, ylabel={$x_2$}, axis 
    equal}}
    \begin{figure}[ht]
        \centering
        \begin{tikzpicture}
            \begin{axis}[custom,xmin=-0.1,xmax=1.25,ymin=-0.15, 
            ymax=1.25,xtick={0,1},ytick={0,1}, legend pos=outer north east]
            \addplot[mark=*,fill=white,only marks,mark size=4pt] 
            coordinates {(0,0)(1,1)};
            \addlegendentry{$g(\vec{x}) = 0$}
            \addplot[mark=*,only marks,mark size=4pt] 
            coordinates {(0,1)(1,0)};
            \addlegendentry{$g(\vec{x}) = 1$}
            \addplot[dashed] {-x+0.5};
            \addplot[dashed] {-x+1.5};
            \addlegendentry{$\psi(\vec{x})$}
            \end{axis}
        \end{tikzpicture}
    \end{figure}

\end{document}